\chapter{Introduction}

\section{Ad-hoc graph Processing}

Many popular graph processing frameworks currently used are all described in reference to an
offline graph processing situation, in which the graph has been built and later it is time to do some
analysis. In this project we propose a graph processing framework designed specifically to perform
ad-hoc graph processing with no preprocessing overhead and adaptive storage. Analysing a graph in real
time as it grows provides the potential to find and exploit insights where it would be too late if
they were found after-the-fact during offline analysis.

Our technique is designed for use with traversal algorithms - algorithms which traverse the graph
by following edges in the course of their computation.

\section{Clustering and Compression}

This project uses database cracking for adaptive clustering. We trial the use of some different
techniques for applying adaptive compression while cracking in order to achieve further performance benefits.

\section{Report Outline}

In the next chapter we will discuss related work in the area of graph processing frameworks and
workload aware frameworks, as well as graph indexing techniques.

After that there will be a chapter explaining and describing the background material required in order
to get the most out of reading this report.

The main body of the work will lie mainly in the chapter on adaptive clustering and compression, which
will include a description of the algorithms we trialled to find the most effective way of performing
adaptive graph queries.

Then there will be chapter on how we evaluated our work, beginning with a description of how the
evaluations were done, including the experimental setup, the algorithms and the datasets. The chapter
concludes with a discussion of our results.

Finally we have a conclusion chapter in which we summarise our contributions and discuss potential future work.

\section{Contributions}

