\chapter{Evaluation}

\label{ch:evaluation}

\section{Experimental Setup}

\section{Graph Algorithms}

\subsection{Breadth First Search}

Breadth first search, or BFS, when run on a graph, involves choosing a starting node as the sole
member of a frontier, which then expands in iterations, wherein members of the frontier add their
out-neighbours which have not yet been visited to the frontier, and remove themselves. This continues
until all vertices have been visited.

There are numerous optimisations which can be made to BFS to improve performance. Storing visited
nodes in a bit-vector provides an effective speed-up, for example. More interestingly, it has been
shown by Beamer et al. that dynamically switching between push and pull iterations can greatly
improve performance on low-diameter, scale free graphs.

The most important factor for us in considering BFS is that it queries the outgoing edges of each
node just once, meaning that the performance gains to be made by an adaptive model are not realised
in the course of a run-through of the algorithm. We use this algorithm to determine what cost there
is to using our adaptive query system when the first query is run, versus a pre-processing system.

\subsection{Pagerank}

Pagerank is a famous algorithm which stores a rank for each vertex and iteratively updates the values
across the entire graph. All nodes are initialised with a rank of $\frac{1}{|V|}$. During each
iteration, each node inherits from all of its in-neighbours a contribution of their pagerank divided
by their out-degree. This value is then multiplied by a damping factor and then added to a base value
to give the updated rank. The base value is defined as $\frac{1 - d}{|V|}$.

In pagerank, every iteration considers all of the vertices and edges in the graph, and so over
an execution of many iterations, any clustering or compression will see further benefits compared to
BFS.

\section{Results}
